\documentclass[12pt]{article}

\usepackage[spanish]{babel}
\usepackage[utf8]{inputenc}
\usepackage{graphicx}
\usepackage{geometry}
\usepackage{verbatim}
\usepackage{xcolor}
\usepackage{fancyhdr}
\usepackage{lastpage}
\usepackage{pdfpages}
\usepackage{listings}
\usepackage{schemata}

\geometry{top=25mm,left=15mm,right=15mm,a4paper}

\pagestyle{fancy}
\fancyhf{}
\lhead{Computación Concurrente}
\cfoot{Página \thepage\ de \pageref{LastPage}}

\graphicspath{./}

\begin{document}
\includepdf{Portada.pdf}
{\color{red} \section*{\textbf{PRACTICA 03}}}
\vspace{1em}

{\color{blue} \subsection*{\textbf{Preguntas:}}}


Deberán detallar a profundidad las respuestas y en caso de ser necesario hacer diagramas 
para ejemplificar tu respuesta.\\


\begin{enumerate}
    \item ¿Qué es un monitor? (2 puntos).
    \vspace{2mm}
    
    \textbf{Un monitor es (objeto sincronizado) una herramienta que se usa para sincronizar el acceso a los recursos compartidos entre procesos o hilos.
    consta de variables de condición y de estado; las de estado contienen información del recurso compartido, las de condición se usan para esperar o notificar
    a otros hilos o procesos sobre cambios en el estado del recurso compartido.}
    \item ¿Qué es un proceso distribuido? (2 puntos).
    \vspace{2mm}
    
    \textbf{Es un tipo de proceso informático que se ejecuta en una red de computadoras interconectadas y que colaboran entre sí 
    para lograr un objetivo común. En un proceso distribuido, cada nodo en la red tiene su propio procesador y memoria, y los nodos se 
    comunican entre sí a través de mensajes para compartir información y coordinar sus acciones.}
    \item ¿Cúales son las diferencias entre Computo Concurrente y Computo Paralelo? (2 puntos).
    \vspace{2mm}
    
    \textbf{Simplemente, El cómputo concurrente se refiere al proceso de ejecutar varias tareas al mismo tiempo en un sistema independientemente de si se ejecutan en diferentes núcleos o procesadores.
    Mientras que el cómputo paralelo se refiere a la ejecución simultánea de tareas en diferentes núcleos o procesadores de un sistema, cada tarea se ejecuta en un procesador o núcleo diferente y puede trabajar en diferentes partes del problema}
    \item Haz un TDA de monitores (4 puntos).
    \vspace{2mm}
    
    \textbf{TDA para monitores:}\\
    \begin{verbatim}
        // Definición del tipo de dato abstracto Monitor

Monitor Monitor {
   // Variables privadas
   int variable1;
   bool variable2;
   // Variables de condición
   CondVar condVar1;
   CondVar condVar2;
   
   // Métodos públicos
   Procedure Inicializar() {
      // Inicialización de variables privadas
      variable1 = 0;
      variable2 = false;
      // Inicialización de variables de condición
      condVar1.Inicializar();
      condVar2.Inicializar();
   }
   
   Procedure Acceder() {
      // Adquirir el monitor
      Monitor.Enter(this);
      // Realizar operaciones en el recurso compartido
      variable1 = 1;
      variable2 = true;
      // Notificar a otros hilos o procesos
      condVar1.Signal();
      // Liberar el monitor
      Monitor.Exit(this);
   }
   
   Procedure Esperar() {
      // Adquirir el monitor
      Monitor.Enter(this);
      // Esperar a que ocurra una condición
      while(!variable2) {
         condVar1.Wait();
      }
      // Realizar operaciones en el recurso compartido
      variable1 = 2;
      // Notificar a otros hilos o procesos
      condVar2.Signal();
      // Liberar el monitor
      Monitor.Exit(this);
   }
   
   Procedure Liberar() {
      // Adquirir el monitor
      Monitor.Enter(this);
      // Realizar operaciones en el recurso compartido
      variable1 = 0;
      variable2 = false;
      // Notificar a otros hilos o procesos
      condVar2.Signal();
      // Liberar el monitor
      Monitor.Exit(this);
   }
}


    
    \end{verbatim}\\
\end{enumerate}

{\color{blue} \subsection*{\textbf{Ejecución:}}}

Para ejecutar nuestro programa es necesario:\\

\begin{enumerate}
    \item Localizarnos en en directorio: \textbf{Practica03}.
    \item Ejecutamos: \textbf{mvn compile}
    \item Ejecutamos el ejecutable que se localiza en el directorio target: \textbf{java -jar target/Practica01.jar}. 
\end{enumerate}
\end{document}